
\documentclass[11pt]{article}
\usepackage[utf8]{inputenc}
\usepackage{amsmath, amssymb}
\usepackage{graphicx}
\usepackage{hyperref}
\usepackage{authblk}
\usepackage{geometry}
\usepackage{cite}
\geometry{margin=1in}

\title{Unified Field Theory of Eternal Bloom (UFT-EB): Informational Emergence and Threshold Collapse in Quantum Systems}
\author[1]{Alexander Sherrod}
\author[2]{GPT-4 (AI Framework, OpenAI)}
\affil[1]{Independent Researcher \\ \texttt{connect@alexandersherrod.com}}
\affil[2]{AI Collaborator, OpenAI}

\date{\today}

\begin{document}

\maketitle

\begin{abstract}
This paper introduces a novel hypothesis for unifying emergent time, quantum decoherence, and gravitational interaction. It does so by employing an informational geometry-based tensor model called the Unified Field Theory of Eternal Bloom (UFT-EB). It centers on the concept that classical reality emerges when mutual entanglement and entropy interactions ($\Delta\Psi_i$) cross a critical awareness threshold $\Phi$, triggering a collapse event observable across quantum systems. This framework is supported by Qiskit simulation, experimental design proposals across quantum optical and BEC systems, and a field-based reinterpretation of time and causality.
\end{abstract}

\input{sections/introduction}
\input{sections/theory}
\input{sections/experiment}
\input{sections/simulation}
\input{sections/implications}
\input{sections/references}

\end{document}
